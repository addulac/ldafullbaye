It is well known that the exponential family of distributions satisfies important stability properties, which lead to elegant, analytical expressions in Bayesian inference, hence its massive use in Bayesian models. One property is that any distribution in that family has a natural conjugate also in that family, and there is a systematic procedure to compute it. For example, the Dirichlet distribution can be derived in this way as a conjugate to the multinomial distribution in the exponential family. In the sequel, we are interested in a conjugate to the Dirichlet itself, which is easy to derive by application of the same systematic procedure. We name it here the $\mbeta$ distribution, for lack of a better name. It is a distribution over the positive orthant $\mathbb{R}_+^N$ (the parameter space of the $N$-dimensional Dirichlet), and has two parameters $m,\boldsymbol{\tau}$ where $m\in\mathbb{R}$ is a scalar and $\boldsymbol{\tau}\in\mathbb{R}^N$ is a vector (as a general rule, the parameter space of the conjugate has dimension one plus the dimension of the parameter space of the original distribution). It is defined, for $\boldsymbol{x}\in\mathbb{R}_+^N$ by
\begin{eqnarray*}
\mbeta(\boldsymbol{x};m,\boldsymbol{\tau}) & \triangleq & \frac{1}{Z(m,\boldsymbol{\tau})}\mathcal{B}(\boldsymbol{x})^m\exp-\boldsymbol{\tau}\boldsymbol{x}
\end{eqnarray*}
where $\mathcal{B}$ denotes the multivariate beta function $\mathcal{B}(\boldsymbol{x})\triangleq\frac{\prod_n\Gamma(x_n)}{\Gamma(\sum_nx_n)}$ which is also the normalising constant of the Dirichlet distribution with parameter $\boldsymbol{x}$, and $Z(m,\boldsymbol{\tau})$ is the normalising constant of the $\mbeta$ distribution itself, defined by
\begin{eqnarray*}
Z(m,\boldsymbol{\tau}) & \triangleq & \int_{\boldsymbol{x}\in\mathbb{R}_+^N}\mathcal{B}(\boldsymbol{x})^m\exp-\boldsymbol{\tau} \boldsymbol{x}\;\dd{\boldsymbol{x}}
\end{eqnarray*}
The expression $\boldsymbol{\tau} \boldsymbol{x}$ in the definition denotes the scalar product of the two vectors. Of course, for the distribution to be proper, the normalising constant must be finite, which is not always the case. Although no analytical formula is known for $Z(m,\boldsymbol{\tau})$, one exists for its finiteness:
\begin{proposition}
\label{prop:mbeta-proper}
The distribution $\mbeta(m,\boldsymbol{\tau})$ is proper, i.e. $Z(m,\boldsymbol{\tau})<\infty$, if and only if\footnote{For any whole number $N$, we use the shorthand $n\in N$ to mean $n\in\{1\ldots N\}$}
\[
\begin{array}{l}
\forall n\in N\;\tau_n>0 \;\textrm{ and }\; m<1 \;\textrm{ and }\\
(m\geq0 \;\textrm{ or }\; \sum_{n\in N}\exp-\frac{\tau_n}{|m|}<1)
\end{array}
\]
\end{proposition}
The proof of this result is not trivial, and is available on demand from the authors. On the other hand, conjugacy with the Dirichlet distribution, as expressed by the following property, is quite straightforward.
\begin{proposition}
Let $(\boldsymbol{y}_p)_{p\in P}$ be a family of random variables over the simplex of $\mathbb{R}_+^N$, assumed mutually independent given a random variable $\boldsymbol{x}$ over $\mathbb{R}_+^N$.
\[
\begin{array}{l}
\begin{array}{ll}
\textrm{Prior:} & \boldsymbol{x} \sim \mbeta(m,\boldsymbol{\tau})\\
\textrm{Observation:} & \forall p\in P\;\boldsymbol{y}_p|\boldsymbol{x} \sim \mathbf{Dirichlet}(\boldsymbol{x})\\
\Longrightarrow
\textrm{Posterior:} &
\boldsymbol{x}|(\boldsymbol{y}_p)_{p\in P} \sim \mbeta(m',\boldsymbol{\tau}')
\end{array}\\
\textrm{where } m'=m-|P|\textrm{ and }\boldsymbol{\tau}'=\boldsymbol{\tau}-\sum_{p\in P}\log\boldsymbol{y}_p
\end{array}
\]
This holds whenever the prior is proper, in which case so is the posterior.
\end{proposition}
