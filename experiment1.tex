%%%%%%%%%%%%%%%%%%%%%%%%%%%%%%%%%%%%%%%%%%%%%%%%%%%%%%%%%%%%%%%%%%%%%%%%%%%
\subsection{A note on perplexity}
%%%%%%%%%%%%%%%%%%%%%%%%%%%%%%%%%%%%%%%%%%%%%%%%%%%%%%%%%%%%%%%%%%%%%%%%%%%
Perplexity is a measure based on the posterior probability of observing a new occurrence of a word $w$ in a document $d$, which is given by
\begin{equation}
\label{eqn:perplex}
\tilde{p}(w|d) =
\sum_k\tilde{p}(w,k|d) =
\sum_k\expectation_{\vmDK_{DK},\vmKW_{KW}\sim\tilde{p}}[\vmDK_{dk}\vmKW_{kw}]
\end{equation}
If we replace the model posterior $\tilde{p}$ by its VB approximation in the expectation, one obtains a product of two marginal expectations since the variables $\vmDK_{dk}\vmKW_{kw}$ are assumed independent in the VB approximation. The marginal expectations are then easy to compute. However, we follow instead a different method, empirically suggested in~\cite{asuncion_smoothing_2009}, and which we justify here. The model posterior expectation in~(\ref{eqn:perplex}) is an arithmetic mean, which is lower bounded by its corresponding geometric mean. The latter factorises in $\vmDK_{dk}\vmKW_{kw}$, even if they are not independent (and they are not, in the model posterior). Now the geometric mean of $\vmDK_{dk}$ (resp. $\vmKW_{kw}$) can be computed in the VB approximation, by simple application of the definition of $\vvmbDK_{dk},\vvmbKW_{kw}$, resulting in
\begin{eqnarray}
\label{eqn:perplex-bound}
\tilde{p}(w|d) & \gtrapprox & \sum_k\exp-(\vvmbDK_{dk}+\vvmbKW_{kw})
\end{eqnarray}
This approximate lower bound, which is also the normaliser in update~$\ruleref{D}$, is used in the definition of the perplexity below (where it becomes an upper bound because of sign change). Up to notations, it is the same term as in Equation (16) of~\cite{asuncion_smoothing_2009}, but it is unclear whether this is the same term as in Equation (16) of~\cite{hoffman_online_2010}.
